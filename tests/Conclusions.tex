%\section{Conclusions}
\label{sec:conclusions}
This paper illustrates the CAMUS framework, whose aim is to empower developers to create context-aware, mobile apps by integrating multiple and heterogeneous APIs acting on situational needs. The  proposed method and its accompanying tools can be the silver bullet for several application domains, as for instance tourism and enterprise. In the tourism domain, CAMUS supports a touristic operator who builds  specialized, context-aware applications for tourists according to their habits, tastes and destinations together with their local points of interest, like museums and restaurants. In the enterprise domain, CAMUS apps act as  personal assistants specifically
tailored for individual managers, allowing them to identify and access in each situation the resources (personal and enterprise data,
business analytics, pieces of news) that must be integrated to support the current decision task. Tourism and enterprise are indeed two domains for which our previous work already highlighted the need of methods and tools for the creation of flexible, situational applications \cite{journals/TWEB2015/CappielloMP15}. 

This paper discusses in particular the major role that our approach gives to context modeling. The specification of the Universal CDT is the central design activity; around it the construction of the mashup is performed. At the design time the designer defines mashup schemas by reasoning at a high level of abstraction on possible context dimensions and associated service categories; at execution time, specific services are dynamically selected and integrated by taking into account the actual user context. The paper also discusses how taking into account context elements in the automatic instantiation of the final app, and especially selecting on the fly the services to be queried is feasible and do not degrade the performance of the server components.

This paper does not discuss the usability of the design method (i.e., how the method is perceived by designers) and the usability of the generated apps (i.e., if they are considered useful and usable by the end users). However, since the CAMUS framework still exploits the composition paradigms and the final app organization that we already defined in our previous work, we capitalize on the large body of data and user feedback collected in the last years through families of user studies (see for example \cite{DBLP:journals/vlc/ArditoCDLMPP14,journals/TWEB2015/CappielloMP15} for an extensive discussion on the conducted evaluations). We are now organizing new evaluation sessions in real development settings where our platform is going to be adopted.
Our future work will be therefore devoted to refining the implementation of the platform, also taking into account the results of the planned studies. 

%%% Local Variables:
%%% mode: latex
%%% TeX-master: "../2016-ICWE"
%%% End:
