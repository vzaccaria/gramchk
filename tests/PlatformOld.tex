\label{sec:platform}
The CAMUS framework is characterized by design environments that, in line with recent approaches to visual programming of mashups, make intensive use of high-level visual abstractions \cite{DBLP:journals/vlc/ArditoCDLMPP14}.

Visual paradigms hide the complexity that is typical of service composition, data integration and mobile application programming, and assist CAMUS designers (even if non-experts in service composition) in the creation of personalized applications that can be run on multiple devices without the need of mastering different technologies.
All the aspects that characterize the different contextual situations, i.e.,
the \emph{dimensions} contributing to context,
are modelled  orthogonally with respect to the other instantaneous system inputs by means of the so-called \emph{Context Dimension Model} \cite{DBLP:journals/cacm/BolchiniCOQRST09,DBLP:journals/debu/BolchiniOQST11}, which provides the constructs to define at design-time
the Universal Context Dimension Tree (\emph{Universal CDT}), i.e., the set of possible contexts of use for a given domain of interest, expressed as a hierarchical structure consisting
of \emph{i)} context dimensions (black nodes), modeling the context variables,
i.e., the different perspectives through which the user perceives the application domain
(e.g., time, place, current company, interest topic), \emph{ii)} the allowed dimension values
(white nodes), i.e., the variable values used to tailor the context-aware information (e.g., ``evening'', ``with friends'', ``music''), and \emph{iii)} variables (e.g., ``geographic coordinates''), that are custom values defined by the user or data acquired by device sensors. Any sub-tree of the CDT with at most a value for each dimension represents a possible user context; this selection is sent to the server with variable values, detected at run-time through device sensors, then a response with arranged data for the given context is returned .
The adoption of a hierarchical
structure allows us to employ different abstraction levels to specify and represent contexts.
%The  designer specifies the context schema at design-time, identifying the possible
%contexts the user may find him/herself in at run-time.

%
\begin{figure} [ht]
\centering
\includegraphics[width= 0.9 \textwidth]{Images/camus-architecture-new2016-medana.pdf}
\caption{Architecture of the CAMUS framework.}
\label{fig:architecture}
\end{figure}
%

Figure \ref{fig:architecture} represents the general organization of the CAMUS framework; it highlights its main architectural components and the flow of the different artifacts that enable the transition from high-level modeling notations to running code.
The framework supports the activities of three main \emph{personae}.

%\begin{itemize}
%\item
The \emph{CAMUS administrator} is in charge of registering distributed resources (remote APIs or in-house services) into the platform, by creating descriptions of how the resources can be invoked and how the responses can be transformed to ensure homogeneity of data formats. S/he also specifies the Universal CDT and the \emph{mappings} between the identified context elements and the parameters accepted by the services previously registered.
%access to information offered by distributed resources (remote APIs or in-house services), and by suggesting proactively services and APIs that can return data of interest with respect to the specified contexts.
In other words, the Universal CDT expresses for each given context a virtual image of the relevant portion of the available resources.

%\item
The \emph{CAMUS (MashUp) designer} starts from the image on the available resources represented by the Universal CDT and, using a \emph{Design Visual Environment}, defines the \emph{Tailored CDT} by further refining the selection of possible contexts based on the needs and preferences of specific users or users' groups. %and the related most appropriate services, also on the basis of the user preferences;

The designer also defines how to mash up the data to be visualized in
the mobile app. The result of this activity is an a JSON-based
\emph{mashup schema}, automatically generated by the design
environment based on a {Domain Specific Language
\cite{journals/TWEB2015/CappielloMP15}}, which includes rules that at run-time guide the execution
of the resulting app.
In addition to this, s/he defines a relation with support services,
where needed, to enrich the user experience (e.g., provide transport
indications to reach a restaurant); they are also context dependent:
for instance, if the user selects ``car'' the system provides route
information, otherwise if s/he selects ``public transport'' it suggests
a bus line.

%,  supported  by providing  them domain-aware access to information collected by the administrator, they tailor the possible contexts that are relevant for a specific user and the most appropriate services, filtered also on the basis of the user preferences.  The designer  is also supported for the specification of APIs composition rules with visual environments.
%\item

The \emph{CAMUS (app) users} are the final recipients of the mobile
app that offers a different bouquet of content and functions in each
different situation of use. When the app is executed, the context
parameter values that characterize the current situation, identified
by means of a client-side \emph{Sensor Wrapper}, are communicated to a
the server; this, in turn, chooses the correct services to be invoked and returns
data in an integrated format. The mashup schema created by the designer is
interpreted {locally} (by means of a \emph{Schema Interpreter}) and the
generated views are populated with the returned data.

The platform indeed exploits generative techniques that comply with
{Model-Driven Engineering methods} modeling abstractions guide the
design of the final applications while generative layers mediate
between high-level visual models and low-level engines that
execute the final mashups. Execution engines, created as {hybrid-native}
applications for different mobile devices, then make it possible the
interpretation and pervasive execution of schemas.

%App content and functionality are adapted according to the context parameters gathered in the current situation of use.
%to transform high-level specifications, defined by the designers through an interactive, visual Web design environment, into running code that executes rules defining both the context-aware selection and composition of the APIs and the context-aware filtering of data retrieved at runtime.
%This means that the notion of context is first used
%at design time, when the  administrator defines, for each context,  a virtual image of the relevant
%portion of the  available resources;  such image is then refined by the designer, who selects the user's possible contexts and the related most appropriate services, also on the basis of the user preferences; at run-time the definition of these context-relevant portions
%will be employed to collect the currently most appropriate APIs and data.

It is worth noting that, in comparison to other approaches to mashup
design \cite{DBLP:books/sp/DanielM14}, the composition activity and
more specifically the selection of services is not exclusively driven
by the functional characteristics of the available services or by the
compatibility of their input and output parameters. Rather, the
initial specification of context requirements enables the progressive
filtering of services first and then the tailoring of service data to
support the final situations of use.


%By exploiting a repository of service descriptors and wrappers used to access local as well as remote  resources, the  \emph{CAMUS Designers}, based on the visual specification of the CDT performed by the administrator,  drives the mashup composition and guides the selection of context-pertinent APIs.
%By means of an intuitive visual notation, the CAMUS Designers are then allowed to specify how the contents coming from such services have to be integrated and visualized within the final app.
%The \emph{CAMUS MashUp Manager} [1] translates the user composition actions into schemas that also include the specification of the context dimensions the final application must be able to react to. Execution engines, created as native applications for different mobile devices, then allows the end users to run the created application by interpreting and executing the schemas pervasively on multiple devices. App content and functionality are adapted according to the context parameters gathered in the current situation of use.

\subsection{Architecture}
%\subsection{Architecture}
\label{subsec:architecture}

The architecture of the final system is server-centric. The framework used to develop the \emph{Server} is Node.js and the database used is MongoDB. The Server's main functionality is to provide the integrated result set to the
mobile app. This process involves \emph{i)} the analysis of the user's context
to select the services to be queried and, \emph{ii)}
querying the selected services and transforming their results into
an integrated data set to be rendered by the mobile app.

%
%\begin{figure} [ht]
%\centering
%\includegraphics[width= 0.6 \textwidth]{Images/camus-architecture-new2016.png}
%\caption{Architecture of the CAMUS framework.}
%\label{fig:architecture}
%\end{figure}
%

The \emph{Server} exposes several endpoints to enable both the execution of service queries as well as CRUD operations on other system data, such as the users, the Universal CDT or the descriptors in the service repository. 
The main API is compliant with the {GraphQL API specification} \cite{docs:specs/facebook/graphql}.
GraphQL offers a layer that enforces a set of custom-defined typing
rules on the data sent and received via HTTP. Besides, it provides a
flexible way to specify the response format, by making it easier to
support different generations of APIs.

The \emph{Visual Design Environment} consists of a suite of Web
applications to: \emph{i)} easily register new services to
the system, \emph{ii)} specify visually (and automatically generate an internal representation of) the CDTs and the
associations of services with pertinent nodes, %(by also identifying filter and ranking
%attributes for service selection), 
and \emph{iii)} design the final mashups and generate their
schema. %The last application provides a virtual device to
%directly see how the final result will be shown on the client. The
%user (of the design environment) can drag and drop the semantic
%definitions (\emph{terms}) that correspond to the fields of the server
%response.   In addition, it is possible to define \emph{support
%services} that can provide additional information to the user.

The \emph{Client App} manages the interaction of the end-user with the whole
system. During its initialization, the app loads  the user CDT and the mashup schemas to be rendered.
The instantiation of views is driven by the mashup schemas and the user interacts
with them in two phases: \emph{i)} when selecting pertinent dimensions characterizing the current context, e.g.,
choosing an interest topic, and \emph{ii)} when accessing the integrated result set built by the server and communicated back to the app. Do note that not all context elements must be chosen by the end user; conversely, most of them will be detected directly by the device, as for example the current place or the temperature.
View rendering is based on a JSON schema file and uses a
cross-platform native technology
to build the resulting user interface elements.
%Most support services are used in the app interacting with the existing
%apps of the system, for instance using the default maps app or the
%default browser.

%
\begin{figure} [t]
\centering
\includegraphics[width=\textwidth]{Images/camus_server_flow_oriz.png}
\caption{Server request flow.}
\label{fig:requestFlow}
\end{figure}
%

%\subsubsection{Request processing steps} 
A typical request from the
client is composed of a JSON payload that describes the \emph{context}
and a specification of the  format of the data that is expected by the
client. As represented in Figure \ref{fig:requestFlow}, the request is thus processed through the following steps:

\begin{itemize}

\item The \emph{Context Manager} parses the context  and ``decorates" it with all the Augmented UCDT information (services, ranks etc.) related to its elements.

\item Based on analyzed context, the \emph{Primary Service Selection} component selects the services to be
queried.

\item The \emph{Query Handler} queries the
selected services by using service-specific bridges that wrap the retrieved result sets and transform them into a common internal representation that complies with the semantic terms associated to the different service attributes. This internal representation enables merging the different data sets based on attributes associated with the same terms.

\item Finally, possible duplicates are removed and the activation of support services - if any, is
bound to the selection of specific attributes in the integrated result set, as defined by the mashup designer when creating the mashup.

\end{itemize}



%%% Local Variables:
%%% mode: latex
%%% TeX-master: "../2016-ICWE"
%%% End:



\subsection{Implementation}
    %\subsection{Implementation}
\label{subsec:implementation}

%In the sequel of this section we describe in detail the role of each component of the system. 

\subsubsection{Primary service selection}
As illustrated above, the \emph{Context
Manager} takes care of  decorating the user's current context with
the information coming from the Augmented  Universal CDT. 
%The context nodes are divided into categories used by
%the \emph{Primary Service Selection} component to select
%the services to be queried.
%This step involves the use of two main categories of nodes, 
%\emph{filter} and \emph{ranking} nodes, as specified during the augmentation of the Universal CDT operated by the platform administrator.
%
Service selection is thus operated by interpreting
the request context as a key-value query, and using this representation to ``traverse'' the Universal CDT specification. 
%The result of the query is a set of primary dimension nodes in the Universal CDT and their associated services. 
The result of the query is a set of \emph{Service Associations} (\emph{SA}) that fit to the current context. Each association is composed by the priority that a service has in the node and the node's weight. These parameters are defined when the Universal CDT is modeled: the priority is an increasing integer starting from 1 and the weights are predefined values assigned following the rule that a filter node's weight must be less than a ranking one.
%Nodes' weights, defined when the Universal CDT is modeled, are considered to compute the relevance of each selected service. These weights depend on the kind of visited nodes: for example, a filter node's weight is minor than a ranking node's weight. 
The final relevance
value for each service \emph{s} is thus computed from the weights $w_i$
and node priority $p_i$ as:

\begin{equation}
R_s = \sum_{i \in SA(s)} \frac{w_i}{p_i}
  \label{eq:relevance}
\end{equation}


The obtained value is used to rank and filter the $N$ top relevant services
for the query.

\subsubsection{Query handling}

%We've built the query handling by decomposing the implementation into
A service-agnostic \emph{Query Handler} composes queries to the selected services; a number of bridges than invoke the services. We supply a default bridge for REST-type services plus an abstract class that can be extended for implementing new
bridges covering further service types. 

A bridge receives a service description
provided by the Query Handler and builds the URL where the
service should be queried. During this composition, the bridge uses the context to
retrieve the list of parameter nodes which, in turn, store the values
that are needed to perform the query.
The bridge also supports the pagination of the result set based on either page number or token. When all the
necessary queries are completed, it sends  the
responses obtained back to the Query Handler.

\subsubsection{Response aggregation}

The \emph{Response Aggregator} executes two main tasks: \emph{i)} merging
items that refer to a same instance and \emph{ii)} scoring each instance.
In fact, two or more services
might return data referring to a same instance; thus duplicate
identification is needed to discover equal or similar instances and merge them
in a unique object. The fusion then produces a richer
set of attributes for an instance, as one service can provide
attributes not supplied by another service. 

%The response returned by the service with the greatest relevance is used as base
%for the merging. As this depends in turn on service priority, it is
%possible for the mashup designer  to influence this operation by
%modifying the priority itself.

Merging is computationally intensive as it requires comparing among them all the instances in any service result set. To reduce this complexity, 
we devised some optimizations. First, instance item is classified by
the phonetic code of its key attribute (for example, the title), using some phonetic string matching metrics\footnote{Our current prototype uses the Chapman's Soundex metrics\cite{Zobel:1996}.}; then, inside each class,
a pair-to-pair comparison of all the common attributes is used to compute a
similarity index. If this value is greater than a predefined
threshold, the two items are considered similar and they are fused
together. The complexity of this comparison strategy is $O(n)$ (i.e., linear in the number of analyzed instances). 

\subsubsection{Support service selection}
The Support Service Selection is similar to the Primary Service
Selection. However, a support service is selected and included in the mashup if and only if all the bindings defined between the mashup core data and the operations exposed by the support service, as defined by the mashup designer, are satisfied. This to
avoid the runtime invocation of services that are not applicable to a
particular context, as the needed input parameters are not provided by the integrated result set or by the usage context. The result of the support service selection
is a set of service endpoints that are communicated to the client within the mashup schema, so that the mobile app can directly invoke the service to retrieve and visualize the auxiliary data.

\subsubsection{Mobile App Execution}

The mobile app is developed using {
React Native} \cite{docs:specs/facebook/react-native}, a tool recently
introduced by Facebook to streamline the production of cross-platform
mobile apps. The app logic is written in Javascript and, for the most
part, is agnostic with respect to the target platform.
React enforces a pseudo-functional/reactive approach that involves a
central state (which holds the \emph{model} of the application) and a
number of pure functions that render the view. The view elements, in
turn, can produce actions that act on the state through a dispatcher
while network responses represent another source of actions that can
change the state.
The state of the app serves the rendering of the views and their
data. It is mainly composed of the mashup data, the current interest
topic, the CDT and the result of the current context-based query.


\emph{\textbf{The app life cycle.}}
At the application startup, the user chooses the current interest
topic. The context selection page allows the editing of the current
context by the user and also probes the hardware for sensors data. When the
user finalizes the context input, a GraphQL query is built and sent to
the server. The request specifies the structure the incoming data
should have in order to be rendered in the results page. An important
difference with respect to a more traditional approach like REST is
that different clients can request different data formats from the same
end point. Once received, the data is stored in the main application
state and the app view is re-rendered by hydrating a React Native
template.

The view schema provides a very flexible mashup design. As reported in Figure \ref{fig:requestFlowApp}, every page is
associated with the corresponding key in the file (e.g.: \texttt{results},
\texttt{details}, ...) and, at render time, the view builder loads the schema
dedicated to the rendering of data for the current topic (tag \texttt{topics}); potentially,
the app is able to render a different view for each possible topic.
The tag \texttt{contents} specifies the view elements; thanks to the
\texttt{style} attribute it is possible to pass directly to the app 
CSS-like style attributes used in React. The elements within the \texttt{contents} tag are
defined recursively, thus enabling a very customizable design of the
app, as in principle any single view element can be defined in this way and then dynamically instantiated.

\begin{figure} [t]
\centering
\includegraphics[width=0.9\textwidth]{Images/app_dataflow_architectureMari.png}
\caption{App data flow.}
\label{fig:requestFlowApp}
\end{figure}

%%% Local Variables:
%%% mode: latex
%%% TeX-master: "../2016-ICWE"
%%% End:


\subsection{Evaluation}
%\subsection{Evaluation}
\label{subsec:evaluation}

In this section we provide a preliminary characterization of the performance of the system.
Since the application is still under active development, the numbers shown here
are to be considered with care. However, we think that they provide some interesting insights on the feasibility of context-aware strategies for service selection and querying, as the ones illustrated in the previous sections. 

\subsubsection{System and workload model.}
To model the system, we use a basic M/G/1 queue \cite{sundarapandian2009probability}. In fact
our system behaves as:

\begin{itemize}
\item M/*/*: a service node where request arrival follows a \emph{markovian}
process, i.e. requests arrive continuously and independently at a
constant average rate $\lambda$. We will use this assumption in the
characterization of the response time.

\item */G/*: the service rate distribution is not yet known, so we assume it being a
general distribution with fixed mean and variance.

\item */*/1: a single process (Node.js) serves incoming requests.
\end{itemize}

The system used for  workload evaluation is characterized by an Intel Core i5-5257U CPU, with 2 cores and a 3GHz frequency, a 8 GB DDR3 RAM, and an SSD disk of 128GB.  

%The following are the characteristics of the system used for the workload evaluation:

%\begin{center}
    %\begin{tabular}{ l l l }
    %Parameter && Value \\ \hline
    %CPU && Intel Core i5-5257U \\
    %Cores && 2 \\
    %CPU frequency && 3GHz \\
    %RAM && 8 GB DDR3 \\
    %Disk && SSD, 128GB \\
    %\end{tabular}
%\end{center}


\subsubsection{Service time.} The service time is the time it takes
for a single request to be served. To better understand the
distribution of the service time (which has been assumed as
\emph{general} in the previous paragraph), we use a sequence of 500
back-to-back requests, where each request is sent once the previous
one has been served. Requests are served by the system with a
first-come/first served (FCFS) policy. We opportunely stubbed the
\emph{query handler} in order to measure just the internal delays of
the system components.

\begin{figure}[t]
\centering
\includegraphics[width=\textwidth]{Images/service-time.pdf}
\caption{Distribution of the service time.}
\label{fig:service-time}
\end{figure}


Figure \ref{fig:service-time} shows the histogram of the measured response time.
To a first inspection, the shape of the distribution seems to agree with a log-normal distribution whose parameters are $\mu = 202 (ms), \sigma = 6.4 (ms)$.
This suggests an ability to sustain almost 5 requests per second. We will use this information to generate a workload of independent requests.

\subsubsection{Response time.}
When receiving independent requests (which can arrive before the current one is effectively served), the system can show a delay which is due to requests queuing up. To characterize the behavior
under this type of workload, we generate a sequence of requests using an exponential arrival-rate distribution. The exponential distribution is in fact congruent with the markovian arrival-rate assumption made above:

$$f(x;\lambda) = \mathrm \lambda e^{-\lambda x}, \textrm{where}~ x \geq 0$$

\noindent where $\lambda$ characterizes the rate of generation of independent requests and $x$ is the
time between one request and the next.

\begin{figure}[t]
\centering
\includegraphics[width=\textwidth]{Images/exponential-response-time.pdf}
\caption{Distribution of the response time under varying workload.}
\label{fig:response-time}
\end{figure}

Figure \ref{fig:response-time} shows the box-plot charts for a varying
request rate, from 1 to 5 requests per seconds (saturation threshold). As can
be seen, the system exhibits a robust response up to $\lambda <
4$. After that point, both variance and mean of the response time
exponentially diverge, approaching the saturation point individuated
in the previous paragraph.

\subsubsection{Discussion.}
The above analysis brings us to an interesting insight which we are going
to investigate further in our work: the service time is log-normally distributed. This type
of distribution is characteristic of a process which is a product of many independent random variables. Our conjecture is that this could be due to the way in which the response elaboration has been split across the components, thus the software composition might play a role in the performance
of the system. This is however a preliminary observation that needs to be corroborated by means of  wider and deeper investigation.

%%% Local Variables:
%%% mode: latex
%%% TeX-master: "../2016-ICWE"
%%% End:


%%% Local Variables:
%%% mode: latex
%%% TeX-master: "../2016-ICWE"
%%% End:
